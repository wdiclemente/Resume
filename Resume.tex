% LaTeX file for resume 
% This file uses the resume document class (res.cls)

\documentclass{res} 
%\usepackage{helvetica} % uses helvetica postscript font (download helvetica.sty)
%\usepackage{newcent}   % uses new century schoolbook postscript font 
\setlength{\textheight}{9.5in} % increase text height to fit on 1-page 

\begin{document} 

\name{\scshape{William K. DiClemente}\\[12pt]}     % the \\[12pt] adds a blank
	
			        % line after name      
\address{{\bf CONTACT INFORMATION}\\will.diclemente@gmail.com\\(414) 617-2645}
\address{{\bf CURRENT ADDRESS}\\3200 Summer St. Unit 5\\Philadelphia, PA 19104}%\\(414) 964-3074\\wdic@sas.upenn.edu}
%\address{\bf PERMANENT ADDRESS \\ 110 Brant Avenue \\  Upper Saddle
%         River,   NJ 07458 \\  (201) 555-9509}
                                  
\begin{resume}

%\section{OBJECTIVE}          
%    A summer teaching position that relates to my physics background.          
 
\section{EDUCATION}          
    {\it University of Pennsylvania, Philadelphia, PA} \\
    Doctor of Philosophy, Physics (Experimental Particle Physics), May 2019 \\
    Masters of Science, Physics, May 2015 \\
    \newline
    {\it Duke University, Durham, NC} \\        
    Bachelor of Science, Physics (High Distinction), May 2013   \\       
    Minors, Mathematics, May 2013 
 
\section{TECHNICAL SKILLS}
    {\bf Programming Languages:} C++, Python\\
    {\bf Data Analysis Frameworks:} ROOT, PyROOT\\
    {\bf Familiar:} Unix-based OS, \LaTeX, Git, Java, Matlab, Fortran

\section{EXPERIENCE}\vspace{-5pt}
   \begin{tabbing}
   \hspace{2.5in}\= \hspace{2.85in} \= \kill % set up two tab positions
    {\bf ATLAS Experiment (CERN)} \>University of Pennsylvania \> 2014-2019\\
    {\it Particle physics researcher} \> Philadelphia, PA \\
    \\
    \> Duke University \> 2010-2013 \\
    \> Durham, NC
   \end{tabbing}\vspace{-18pt}
   Physics research with the ATLAS Collaboration at the CERN Large Hadron Collider (LHC) as a graduate and undergraduate student.
   Research involves using and writing analysis software to read through the large volumes of data collected by the experiment.
   \begin{itemize}
     \item Physics analysis (2011-2013, 2015-2019): Analysis of LHC data to identify and measure particular particle interactions, such as electroweak bosons, and compare to theoretical predictions.
     \item Detector performance (2014-2019): Maintain high measurement quality by deriving and applying software-level corrections to account for physical movements of detector elements.
   \end{itemize}

   \vspace{-0.1in}	
   \begin{tabbing}
     \hspace{2.5in}\= \hspace{2.9in}\= \kill % set up two tab positions
    {\bf Physics Lab} \>University of Pennsylvania     \> 2013-2014\\
    {\it Teaching Assistant}  \>Philadelphia, PA
   \end{tabbing}\vspace{-18pt}      % suppress blank line after tabbing
   Taught the laboratory component of the undergraduate introductory physics courses: three mechanics and two electricity and magnetism labs across three semesters.
   Responsibilities included demos and guidance, teaching necessary content if not covered in lecture, lab report grading.

\section{HOBBIES \& INTERESTS}
  \begin{itemize}
    \item Cello
    \item IM sports
    \item BoI sreams
  \end{itemize}

\section{PUBLICATIONS}
    DiClemente, William K., {\it Measurement of Electroweak Production of Same-Sign W Boson Pairs with ATLAS}.  PhD thesis, May, 2019. {\tt http://cds.cern.ch/record/2674035}. Presented 21 Feb, 2019.

    ATLAS Collaboration, {\it Observation of electroweak production of a same-sign WW boson pair in association with two jets in pp collisions at $\sqrt{s}=13~\textrm{TeV}$ with the ATLAS detector}. Submitted to Phys. Rev. Lett. June 2019. {\tt arXiv}:1906.03203 [hep-ex].

    ATLAS Collaboration, {\it Prospects for the measurement of the $W^{\pm}W^{\pm}$ scattering cross section and extraction of the longitudinal scattering component in pp collisions at the High-Luminosity LHC with the ATLAS experiment}. CERN, Geneva, Dec, 2018. {\tt http://cds.cern.ch/record/2652447}. %(Included in {\it Report on the Physics at the HL-LHC and Perspectives for the HE-LHC}, CERN, March, 2019.)

    ATLAS Collaboration, {\it Measurement of the $W^{\pm}Z$ boson pair-production cross section in pp collisions at $\sqrt{s}=13~\textrm{TeV}$ with the ATLAS detector}. Phys. Lett. {\bf B}762 (2016) 1-22, {\tt arXiv}:1606.04017 [hep-ex].
 
%\section{HONORS AND AWARDS}          
%    Honors and awards go here
 
%\section{EXTRACURRICULAR ACTIVITIES}          
%    extracurriculars here
 
\end{resume}
\end{document}
