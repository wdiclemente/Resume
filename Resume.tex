% LaTeX file for resume 
% This file uses the resume document class (res.cls)

\documentclass[10pt]{res}
%\topmargin=-0.25in
\newsectionwidth{20pt} % reduce the indent in the sections
%\usepackage{helvetica} % uses helvetica postscript font (download helvetica.sty)
%\usepackage{newcent}   % uses new century schoolbook postscript font 
\usepackage{amsmath}
\renewcommand\labelitemi{--}
\setlength{\textheight}{9.5in} % increase text height to fit on 1-page 

\begin{document} 

\name{\scshape{William K. DiClemente, PhD}\\[12pt]}     % the \\[12pt] adds a blank
			        % line after name      
\address{}
\address{\hspace{31pt}wdic@sas.upenn.edu\\\hspace{55pt}(414) 617-2645\\LinkedIn: {\tt bit.ly/2lzZi0U}}
%\address{www.linkedin.com/in/will-diclemente/}%\\(414) 964-3074\\wdic@sas.upenn.edu}
%\address{\bf PERMANENT ADDRESS \\ 110 Brant Avenue \\  Upper Saddle
%         River,   NJ 07458 \\  (201) 555-9509}
                                  
\begin{resume}

%\section{OBJECTIVE}          
%    A summer teaching position that relates to my physics background.          
 
\section{EDUCATION}          
    {\bf University of Pennsylvania}, Philadelphia, PA \\
    {\bf PhD}, Experimental Particle Physics, May 2019 \\
    Masters of Science, Physics, May 2015 \\
    \newline
    {\bf Duke University}, Durham, NC \\        
    Bachelor of Science, Physics (High Distinction), May 2013   \\       
    Minors, Mathematics, May 2013 
 
\section{TECHNICAL SKILLS}
    {\bf Proficient in} C++, Python, ROOT/PyROOT (Data analyis framework)\\
    {\bf Experienced in} Unix-based OS, \LaTeX, MySQL, Bash, Git, Java %, Matlab, Fortran

\section{RESEARCH EXPERIENCE}%\vspace{-10pt}
    {\bf Particle Physics Research with the ATLAS Experiment at CERN}\\
    University of Pennsylvania (2014-2019)/Duke University (2010-2013)

    As a physics researcher, I used a combination of C++ and Python software including ROOT, experiment-wide frameworks, and personal analysis-specific software to read, analyze, and visualize terabytes of real and simulated ATLAS data.
    My research was highly collaborative; our analysis teams would regularly report progress with parent groups, interact with experts on detector performance, and consult with theorists for additional ideas and models to test.
    
    % large data set -- terabytes (petabytes) in size
    % existing analysis frameworks in C++ and python
    % write own code in C++ and python, analyze/visualize with ROOT
    % problem solving -> top background too high, update fake factor, etc
    % leading role in WW upgrade, leading in fake factor method
    % Collaborative, worked in analysis group and performance group (talk to other analyzers within or outside of your group, performance people, theorists)

    {\bf Research highlights} include:
    \begin{itemize}
    %\item Played a leading role in an analysis using simulated data for a proposed future collider experiment.
    \item Played a leading role in the development of an updated technique for modeling troublesome background processes in a high-profile physics analysis.
    \item Slimmed and skimmed many-terabyte datasets into smaller, analysis-specific samples for several different analyses.
    \item Optimized an analysis's signal selection using a random grid search algorithm, improving the significance by nearly 60\%.
    \item Introduced a new set of 2D cuts to an analysis which reduced a major background by 20\%.
    \item Monitored detector performance for possible biases in data reconstruction using 2D maps built from fits to distributions of various measured quantities.
    \item Analysis work resulted in 4 papers, as well as being a contributing author on over 100 additional ATLAS publications.
    \end{itemize}

\section{TEACHING EXPERIENCE}%\vspace{-10pt}
    {\bf Introductory Physics Laboratory Teaching Assistant}\\
    University of Pennsylvania (2013-2014)

    Taught three semesters of classical mechanics and electricity and magnetism labs. Responsibilities included demonstrating lab techniques, guiding students through their exercises, teaching supplemental material, and grading lab reports.
%%%%%
% stuff from CV, saved for reference
%%%%%
%   ATLAS is one of the particle detector experiments at CERN's Large Hadron Collider (LHC) in Geneva, Switzerland. 
%   It is one of the largest scientific collaborations ever, consisting of over 3000 scientists stationed around the world and at CERN.
%   
%   Nearly ten years of research experience as a graduate and undergraduate student including physics measurements and detector performance studies.
%   Projects typically are collaborative with a team of scientists working in parallel to complete the study.
%   Research involves using ATLAS's data analysis framework as well as writing personal or project-specific analysis software to process the large volumes of data collected by the detector.
%   \begin{itemize}
%     \item {\it Physics analysis} (2011-2013, 2015-2019): Analysis of LHC collision data to identify and measure particle interactions, such as electroweak boson scattering, and compare to theoretical predictions.
%           Candidate events passing specific signal criteria are selected and various background processes are modeled in order to measure the process of interest.
%     \item {\it Detector performance} (2014-2019): High measurement quality is essential for precise physics measurements and is maintained through studies of ATLAS's performance.
%           Physical movements of detector sensors that can occur during normal operation are accounted for by deriving and applying corrections at software-level when data is processed.
%   \end{itemize}

%   \vspace{-0.1in}	
%   \begin{tabbing}
%     \hspace{2.5in}\= \hspace{2.9in}\= \kill % set up two tab positions
%    {\bf Physics Lab} \>University of Pennsylvania     \> 2013-2014\\
%    {\it Teaching Assistant}  \>Philadelphia, PA
%   \end{tabbing}\vspace{-18pt}      % suppress blank line after tabbing
%   Taught the laboratory component of the undergraduate introductory physics courses for classical mechanics and electricity and magnetism across three semesters.
%   Responsibilities included lab demos and guidance, teaching necessary content if not covered in lecture, and lab report grading.

%\newpage
%\section{HOBBIES \& INTERESTS}
%   Played cello for nearly twenty years, was a member of the local youth orchestra in high school and the university orchestra in college.
%
%   Participated in a variety of intramural sports in college and graduate school including flag football, basketball, and ultimate frisbee.
%
%   Hosted and provided commentary for a number of speedrunning tournaments on Twitch for the game The Binding of Isaac as well as speedrunning it myself.

%\section{SELECTED PUBLICATIONS}
%    DiClemente, William K., {\it Measurement of Electroweak Production of Same-Sign W Boson Pairs with ATLAS}.  PhD thesis. {\tt http://cds.cern.ch/record/2674035}. Presented 21 Feb, 2019.

%    ATLAS Collaboration, {\it Observation of electroweak production of a same-sign WW boson pair in association with two jets in pp collisions at $\sqrt{s}=13~\textrm{TeV}$ with the ATLAS detector}. Submitted to Phys. Rev. Lett. June 2019. {\tt arXiv}:1906.03203 [hep-ex].
%%%
% ``list at most two papers''
%%%
%    ATLAS Collaboration, {\it Prospects for the measurement of the $W^{\pm}W^{\pm}$ scattering cross section and extraction of the longitudinal scattering component in pp collisions at the High-Luminosity LHC with the ATLAS experiment}. CERN, Geneva, Dec, 2018. {\tt http://cds.cern.ch/record/2652447}. %(Included in {\it Report on the Physics at the HL-LHC and Perspectives for the HE-LHC}, CERN, March, 2019.)

%    ATLAS Collaboration, {\it Measurement of the $W^{\pm}Z$ boson pair-production cross section in pp collisions at $\sqrt{s}=13~\textrm{TeV}$ with the ATLAS detector}. Phys. Lett. {\bf B}762 (2016) 1-22, {\tt arXiv}:1606.04017 [hep-ex].
 
%\section{HONORS AND AWARDS}          
%    Honors and awards go here
 
%\section{EXTRACURRICULAR ACTIVITIES}          
%    extracurriculars here
 
\end{resume}
\end{document}
