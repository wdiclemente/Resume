% LaTeX file for resume 
% This file uses the resume document class (res.cls)

\documentclass{res} 
%\usepackage{helvetica} % uses helvetica postscript font (download helvetica.sty)
%\usepackage{newcent}   % uses new century schoolbook postscript font 
\setlength{\textheight}{9.5in} % increase text height to fit on 1-page 

\begin{document} 

\name{\scshape{William K. DiClemente}\\[12pt]}     % the \\[12pt] adds a blank
	
			        % line after name      
\address{{\bf CONTACT INFORMATION}\\will.diclemente@gmail.com\\(414) 617-2645}
\address{{\bf CURRENT ADDRESS}\\3200 Summer St. Unit 5\\Philadelphia, PA 19104}%\\(414) 964-3074\\wdic@sas.upenn.edu}
%\address{\bf PERMANENT ADDRESS \\ 110 Brant Avenue \\  Upper Saddle
%         River,   NJ 07458 \\  (201) 555-9509}
                                  
\begin{resume}

%\section{OBJECTIVE}          
%    A summer teaching position that relates to my physics background.          
 
\section{EDUCATION}          
    {\it University of Pennsylvania, Philadelphia, PA} \\
    Doctor of Philosophy, Physics (Experimental Particle Physics), May 2019 \\
    Masters of Science, Physics, May 2015 \\
    \newline
    {\it Duke University, Durham, NC} \\        
    Bachelor of Science, Physics (High Distinction), May 2013   \\       
    Minors, Mathematics, May 2013 
 
\section{TECHNICAL SKILLS}
    {\bf Programming Languages:} C++, Python\\
    {\bf Data Analysis Frameworks:} ROOT, PyROOT\\
    {\bf Familiar:} Unix-based OS, \LaTeX, Git, Java, Matlab, Fortran

\section{EXPERIENCE}\vspace{-5pt}
   \begin{tabbing}
   \hspace{2.5in}\= \hspace{2.85in} \= \kill % set up two tab positions
    {\bf ATLAS Experiment (CERN)} \>University of Pennsylvania \> 2014-2019\\
    {\it Particle physics researcher} \> Philadelphia, PA \\
    \\
    \> Duke University \> 2010-2013 \\
    \> Durham, NC
   \end{tabbing}\vspace{-18pt}
   ATLAS is one of the particle detector experiments at CERN's Large Hadron Collider (LHC) in Geneva, Switzerland. 
   It is one of the largest scientific collaborations ever, consisting of over 3000 scientists stationed around the world and at CERN.
   
   Nearly ten years of research experience as a graduate and undergraduate student including physics measurements and detector performance studies.
   Projects typically are collaborative with a team of scientists working in parallel to complete the study.
   Research involves using ATLAS's data analysis framework as well as writing personal or project-specific analysis software to process the large volumes of data collected by the detector.
   \begin{itemize}
     \item {\it Physics analysis} (2011-2013, 2015-2019): Analysis of LHC collision data to identify and measure particle interactions, such as electroweak boson scattering, and compare to theoretical predictions.
           Candidate events passing specific signal criteria are selected and various background processes are modeled in order to measure the process of interest.
     \item {\it Detector performance} (2014-2019): High measurement quality is essential for precise physics measurements and is maintained through studies of ATLAS's performance.
           Physical movements of detector sensors that can occur during normal operation are accounted for by deriving and applying corrections at software-level when data is processed.
   \end{itemize}

   \vspace{-0.1in}	
   \begin{tabbing}
     \hspace{2.5in}\= \hspace{2.9in}\= \kill % set up two tab positions
    {\bf Physics Lab} \>University of Pennsylvania     \> 2013-2014\\
    {\it Teaching Assistant}  \>Philadelphia, PA
   \end{tabbing}\vspace{-18pt}      % suppress blank line after tabbing
   Taught the laboratory component of the undergraduate introductory physics courses for classical mechanics and electricity and magnetism across three semesters.
   Responsibilities included lab demos and guidance, teaching necessary content if not covered in lecture, and lab report grading.

\newpage
\section{HOBBIES \& INTERESTS}
   Played cello for nearly twenty years, was a member of the local youth orchestra in high school and the university orchestra in college.

   Participated in a variety of intramural sports in college and graduate school including flag football, basketball, and ultimate frisbee.

   Hosted and provided commentary for a number of speedrunning tournaments on Twitch for the game The Binding of Isaac as well as speedrunning it myself.

\section{PUBLICATIONS}
    DiClemente, William K., {\it Measurement of Electroweak Production of Same-Sign W Boson Pairs with ATLAS}.  PhD thesis, May, 2019. {\tt http://cds.cern.ch/record/2674035}. Presented 21 Feb, 2019.

    ATLAS Collaboration, {\it Observation of electroweak production of a same-sign WW boson pair in association with two jets in pp collisions at $\sqrt{s}=13~\textrm{TeV}$ with the ATLAS detector}. Submitted to Phys. Rev. Lett. June 2019. {\tt arXiv}:1906.03203 [hep-ex].

    ATLAS Collaboration, {\it Prospects for the measurement of the $W^{\pm}W^{\pm}$ scattering cross section and extraction of the longitudinal scattering component in pp collisions at the High-Luminosity LHC with the ATLAS experiment}. CERN, Geneva, Dec, 2018. {\tt http://cds.cern.ch/record/2652447}. %(Included in {\it Report on the Physics at the HL-LHC and Perspectives for the HE-LHC}, CERN, March, 2019.)

    ATLAS Collaboration, {\it Measurement of the $W^{\pm}Z$ boson pair-production cross section in pp collisions at $\sqrt{s}=13~\textrm{TeV}$ with the ATLAS detector}. Phys. Lett. {\bf B}762 (2016) 1-22, {\tt arXiv}:1606.04017 [hep-ex].
 
%\section{HONORS AND AWARDS}          
%    Honors and awards go here
 
%\section{EXTRACURRICULAR ACTIVITIES}          
%    extracurriculars here
 
\end{resume}
\end{document}
